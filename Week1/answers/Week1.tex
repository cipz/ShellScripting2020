% !TeX program = pdflatex
% !TEX encoding = UTF-8 Unicode

\documentclass[9pt]{article}
\usepackage{hyperref}
\usepackage{ulem} 
\usepackage{mathtools}
\usepackage{enumitem}
\usepackage{color,soul}
\usepackage{inputenc}
\usepackage[margin=2.75cm]{geometry}
\usepackage{xcolor}
\usepackage{listings}

% TODO Modificare questi parametri per scrivere un bash più bello

\lstset{basicstyle=\ttfamily,
	showstringspaces=false,
	commentstyle=\color{blue},
	keywordstyle=\color{red}
}
\lstset{
	language=bash,
	basicstyle=\ttfamily
}

\begin{document}

\title{Shell Scripting 2020: Week 1}
\author{Stefan Ciprian Voinea}
\maketitle

%\begin{figure}[h!]
%	\centering
%	\includegraphics[width=12cm]{autoconfiguration.png}
%	\caption{IPv6 Autoconfiguration example}
%	\label{fig:autoconfig}
%\end{figure}

\begin{enumerate}
	\item \textbf{Directory setup}
		\begin{lstlisting}[language=bash]
mkdir ShellScripting2019
cd ShellScripting2019
mkdir Week1
cd Week1
		\end{lstlisting}

	\item \textbf{Identity shift}
		\begin{itemize}
			\item alias for \texttt{ls}:
			\begin{lstlisting}[language=bash]
alias show-files-here=ls
			\end{lstlisting}
			\item for the definition of \texttt{cman} command I wrote a simple script that opens firefox and a website that contains the manual of linux commands:
			\begin{lstlisting}[language=bash]
url='https://man.cx/'
firefox $url$1 
			\end{lstlisting}
			I set the alias with this command:
			\begin{lstlisting}[language=bash]
alias cman=./task2.sh
			\end{lstlisting}
		\end{itemize}


	% TODO DA FARE ALL'INTERNO DEL DIPARTIMENTO O DA CASA CON SSH
	\item \hl{\textbf{(NON)-logins}}
		\begin{lstlisting}[language=bash]
		
		\end{lstlisting}

	% TODO DA FARE ALL'INTERNO DEL DIPARTIMENTO O DA CASA CON SSH
	\item \hl{\textbf{Enter RSYNC}}
		\begin{lstlisting}[language=bash]
		\end{lstlisting}

	\item \textbf{Time and Date}\\
		This is the command:
		\begin{lstlisting}[language=bash]
date +%A.%Y.%m.%d 
		\end{lstlisting}
		And this is the output:
		\begin{lstlisting}[language=bash]
Friday.2020.10.30
		\end{lstlisting}
		
	\item \textbf{Inserting date}
		The command that is contained in the file \texttt{task6.sh}:
		\begin{lstlisting}[language=bash,breaklines=true]
echo "rsync --archive /cs/home/tkt_cam/public_html/$(date +%Y/%m/%d)/~/ShellScripting2019/Week1/$(date +%A.%Y.%m.%d)"
		\end{lstlisting}
		
	\item \textbf{Two at once}\\
		By using \texttt{command >out 2>error} I can direct the \texttt{stdout} to the file \texttt{out} and the \texttt{stderr} to the file \texttt{error}.\\
		This command works, does not show any output because it is written in the \texttt{out} file.
		\begin{lstlisting}[language=bash]
ls >out 2>error
		\end{lstlisting}
		This command doesn't works because the \texttt{ghost\_directory} does not exist.
		There is no output output because it is written in the \texttt{error} file.
		\begin{lstlisting}[language=bash]
ls ghost_directoy >out 2>error
		\end{lstlisting}
		Content on error file:
		\begin{lstlisting}[language=bash]
ls: cannot access 'ghost_directory': No such file or directory
		\end{lstlisting}
		
	\item \textbf{Hey! What about STDIN?}\\
		As explained by the command \texttt{man cat}: \texttt{cat - concatenate files and print on the standard output}.
		This means that if I type \texttt{cat string}, it will print my string when I press enter, while if I type \texttt{cat filename}, it will print the contents of the file.
	
\end{enumerate}

\end{document}