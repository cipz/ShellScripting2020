% !TeX program = xelatex
% !TEX encoding = UTF-8 Unicode

\documentclass[9pt]{article}
\usepackage{hyperref}
\usepackage{ulem} 
\usepackage{mathtools}
\usepackage{enumitem}
\usepackage{color,soul}
\usepackage{inputenc}
\usepackage[margin=2.75cm]{geometry}
\usepackage{xcolor}
\usepackage{listings}

% TODO Modificare questi parametri per scrivere un bash più bello

\lstset{basicstyle=\ttfamily,
	showstringspaces=false,
	commentstyle=\color{blue},
	keywordstyle=\color{red}
}
\lstset{
	language=bash,
	basicstyle=\ttfamily
}

\begin{document}

\title{Shell Scripting 2020: Week 2}
\author{\textbf{Stefan Ciprian Voinea}\\Student number: 015383372}
\maketitle

%\begin{figure}[h!]
%	\centering
%	\includegraphics[width=12cm]{autoconfiguration.png}
%	\caption{IPv6 Autoconfiguration example}
%	\label{fig:autoconfig}
%\end{figure}

\begin{enumerate}
	
	\setcounter{enumi}{8}
	
	\item \textbf{Pipelines}
	
		Contents of the \texttt{task9.sh} file:
		\begin{lstlisting}
# Counting all files 
ls ~ | tee ls_output
wc -l ls_output

# Counting only folders
ls -d */ | tee ls_output_dirs
wc -l ls_output
		\end{lstlisting}

		Alternatively I could have used the command \texttt{ls dir > ls\_output}.
		
	\item \textbf{Filters}

		Contents of the \texttt{task10.sh} file:
		\begin{lstlisting}
ls ~ | grep 'e' | wc -l
		\end{lstlisting}

		Output of the execution:

		\begin{lstlisting}
$ ./task10.sh 
5
		\end{lstlisting}
		
	\item \textbf{Interlude: Bash}
	
	Contents of the \texttt{count-homedir.sh} file:
	\begin{lstlisting}
ls -p ~ | egrep -v /$ | wc -l
	\end{lstlisting}

	\item \textbf{Some Assembly required}
	
		Contents of the \texttt{task12\_list\_files\_subdirectories\_November\_2011.sh} file:

		\begin{lstlisting}[language=bash,breaklines=true]
# Syncing with remote directory
rsync --archive /cs/home/tkt_cam/public_html/2011/11 ~/Desktop/ShellScripting2020/Week2/2011 --stats

# Listing files, directories and subdirectories
ls -R ~/Desktop/ShellScripting2020/Week2/2011/11
		\end{lstlisting}
	
	\item \textbf{Now we want only the pictures}
	
		Contents of the \texttt{task13\_list\_all\_jpg\_November\_2011.sh} file:
		\begin{lstlisting}[language=bash,breaklines=true]
# Syncing with remote directory
rsync --archive /cs/home/tkt_cam/public_html/2011/11 ~/Desktop/ShellScripting2020/Week2/2011 --stats

# Listing files, directories and subdirectories
ls -R ~/Desktop/ShellScripting2020/Week2/2011/11 | grep jpg
		\end{lstlisting}
		
		Alternatively I could have simply called the previous script like this:
		\begin{lstlisting}[language=bash,breaklines=true]
task12_list_files_subdirectories_November_2011.sh | grep jpg               
		\end{lstlisting}

	\item \textbf{How many do we have so far?}
	
		Contents of the \texttt{task14\_count\_picture\_files.sh} file:
		\begin{lstlisting}[language=bash,breaklines=true]
# Syncing with remote directory
rsync --archive /cs/home/tkt_cam/public_html/2011/11 ~/Desktop/ShellScripting2020/Week2/2011 --stats

# Listing files, directories and subdirectories
ls -R ~/Desktop/ShellScripting2020/Week2/2011/11 | grep jpg | wc -l
		\end{lstlisting}

		Alternatively I could have simply called the previous script like this:
		\begin{lstlisting}[language=bash,breaklines=true]
wc -l task13_list_all_jpg_November_2011.sh
		\end{lstlisting}
	
	\item \textbf{Remember the backticks}
	
		Contents of the \texttt{task15\_list\_all\_jpg\_current\_month.sh} file:
		\begin{lstlisting}[language=bash,breaklines=true]
# Syncing with remote directory
# echo `rsync --archive /cs/home/tkt_cam/public_html/$(date +%Y/%m) ~/Desktop/ShellScripting2020/Week2/$(date +%Y) --stats`
rsync --archive /cs/home/tkt_cam/public_html/2016/11 ~/Desktop/ShellScripting2020/Week2/2016/11 --stats

# Listing files, directories and subdirectories
# echo `ls -R ~/Desktop/ShellScripting2020/Week2/$(date +%Y/%m) | grep jpg | wc -l`
ls -R ~/Desktop/ShellScripting2020/Week2/2016/11 | grep jpg | wc -l
		\end{lstlisting}

	\item \textbf{The big brother of \texttt{ls}}
	
		Contents of the \texttt{task16\_find\_image\_files\_corresponding\_current\_month\_and\_count.sh} file:
		\begin{lstlisting}[language=bash,breaklines=true]
# Syncing with remote directory
# echo `rsync --archive /cs/home/tkt_cam/public_html/$(date +%Y/%m) ~/Desktop/ShellScripting2020/Week2/$(date +%Y) --stats`
rsync --archive /cs/home/tkt_cam/public_html/2016/11 ~/Desktop/ShellScripting2020/Week2/2016/11 --stats

# Listing image files
# find ~/Desktop/ShellScripting2020/Week2/$(date +%Y/%m) -name '*.jpg' -type f | wc -l
find ~/Desktop/ShellScripting2020/Week2/2016/11 -name '*.jpg' -type f | wc -l
		\end{lstlisting}

\end{enumerate}

\end{document}